% uplatex sx1280
% uplatex sx1280
% dvipdfmx sx1280

\documentclass[a4j,oneside]{ujbook}
\usepackage[schinese,japanese]{pxbabel} % \foreignlanguage{schinese}{}
\usepackage[dvipdfmx]{graphicx} % \includegraphics{}
\usepackage{bmpsize}
\usepackage[dvipdfmx]{hyperref} % \url{}
\usepackage{pxjahyper} % 日本語しおり ※hyperref → pxjahyper の順であること
\usepackage{enumitem} % \begin{description}[nextline]
\usepackage{ascmac} % \yen
\usepackage{here} % [H]
\begin{document}

\title{no title}
\author{no name}
\date{\today}

\chapter*{第 6 送信機系統図}

\setlength{\unitlength}{4mm}
\begin{picture}(35,25)
\put(0,13){\dashbox{0.125}(8,4){\begin{tabular}{l}{電子計算機 (パーソ}\\{ナルコンピュータや}\\{スマートフォン等)}\end{tabular}}}
\put(9,15){\vector(-1,0){1}}
\put(9,15){\vector(1,0){1}}
\put(10,10){\framebox(8,8){\begin{tabular}{l}{MCU MD-328D}\\{※ ATmega328/}\\{LGT8F328 等}\\{互換 MCU の}\\{場合あり}\end{tabular}}}
\put(19,15){\vector(-1,0){1}}
\put(19,15){\vector(1,0){1}}
\put(20,13){\framebox(8,4){\begin{tabular}{l}{Transceiver}\\{SX1280}\end{tabular}}}
\put(24,11){\vector(0,1){2}}
\put(22,7){\framebox(4,4){\begin{tabular}{l}{Xtal}\\{52MHz}\end{tabular}}}

\put(19,5){\begin{tabular}{l}{Ebyte E28-2G4M12S}\\{LoRa Module}\end{tabular}}
\put(19,3){\dashbox{0.125}(10,15)}

\put(10,3){第 6 送信機}
\put(9,2){\dashbox{0.25}(21,17)}

\put(28,15){\line(1,0){3}}
\put(31,15){\line(0,1){5}}
\put(31,14){2427 〜 2450 MHz}

\put(31,20){\line(0,1){2}}
\put(31,20){\line(-1,2){1}}
\put(31,20){\line(1,2){1}}
\put(30,22){\line(1,0){2}}
\put(32,19){ANT}

\end{picture}

\vspace{1cm}
\begin{tabular}{ll}
 {諸元} \\
 \hline
 {電波型式} & {F1D} \\
 {通信速度} & {297.55 〜 203125 bps} \\
 {変調方式} & {LoRa 変調 (チャープ・スペクトラム拡散変調)} \\
 {占有周波数帯域幅} & {203.125 〜 1625.0 kHz} \\
 {拡散率} & {\( 2^{SF} \) (SF = 5 〜 12)} \\
 {誤り訂正符号} & {ハミング符号 (5 〜 8, 4)} \\
 {符号構成} & {LoRa 規格準拠}
\end{tabular}

\vspace{1cm}
※ E28-2G4M12S 制御用ソフトウェアの概要については別紙に記す

\end{document}
